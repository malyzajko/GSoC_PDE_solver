\documentclass[12pt]{article}
\usepackage{amsmath}
\usepackage{amsfonts}
\usepackage[margin=1in]{geometry}

\usepackage{enumerate}
\title{PDE Solver Mid-Term Report}
\date{}
\begin{document}
\maketitle
\newcommand{\expr}{{\ttfamily Expr}}
\newcommand{\code}[1]{{\ttfamily #1}}

\section{Week by Week Summary}
This is an approximate week by week summary of the work which has been
done thus far. Since this summary was written after 5 weeks of work
based on memory, some past details may be innacurate. 
\begin{itemize}
    
    
\item[June 17th:] \begin{itemize}
        
    \item Began the project by implementing a solver for elliptical
        2nd order bivariate pdes. The pdes we're initially stored as a
        series of functions, similarly for the boundaries.
    \item This usage was very unatural, so I was advised to change
        it. The code written during this week used later on with
        almost no modificaitons.
    \end{itemize}
    
    
\item[June 24th:]\begin{itemize}
        
    \item Most of this week was spent designing the interface and structure of the
        solver.

    \item I considered making a parser to convert strings to pde
        objects, but a DSL seemed more effective. 
        
    \item Instead of storing a pde as a set of functions, it is now a
        set of \expr's (Expressions). \expr's are abstract syntax
        trees representing algebraic expressions.
        
    \item This also allowed for the distinction between function
        variables (\code{FunctionVariable}) and other variables
        (\code{NonFunctionVariable}).

    \item The syntax for a pde would be something like:
        
        \code{val laplace = dd(u, x, x) + dd(u, t, t) === 0}
        
        where \code{u} has already been defined as a function variable,
        and \code{x} and \code{t} have already been set as
        non-function variables.
        \code{dd} and \code{d} are classes used to represent
        derivatives. \code{===} is an \expr method which takes in as
        input another \expr, in this case \code{Const(0)}, and spits
        out a pde.
        
    \item The code for most expression and variable code was all
        written this week.
    \end{itemize}

\item[July 1st:]\begin{itemize}

    \item Implemented methods to evaluate expression. The method takes
        as input a map from non-function variables and evaluates the
        syntax tree as needed.

    \item Redesigned and wrote the code for boundary
        functions. Initially a boundary was also defined through some
        awkward combination of functions and tuples which depended
        heavily on the order in which the functions we're passed in.

        A rectangular boundary is now composed of multiple
        \code{BFunction}'s (Boundary Functions). These would look like
        
        \code{new BFunction(fixVar(x, 0), From(0, 10), condition(u,
            0))}

        in math this would look like
        \begin{align*}
            u(0, t) & = 0& 0<t<10
        \end{align*}

    \item Implemented the code to split up an expression into it's
        derivative terms and then convert it to a pde. 
    \end{itemize}

    
\item[July 8th:]\begin{itemize}

    \item Started implementing a data type for values with error
        associated with them. I decided to read up more about similar
        implementations, but I forgot about them while working on
        other parts of the code.
        
    \item Reimplemented the code for solving elliptical equations.
        Most of it was exactly the same as done in the first week. For
        now what's returned is a tuple of 2 2D arrays of doubles. One
        for the solution and one for the error associated with each
        point.

    \item Tested the solver with Laplace's equation with fairly simple
        boundary conditions. Solving for Laplace's equation went
        well with no notable problems.  

    \item The boundary assertions had to be loosened for approximately
        equals due to truncation differences between doubles. 
    \end{itemize}
    
\item[July 15th:]\begin{itemize}

    \item For clarity sake, the constructor for a pde no longer takes
        in multiple expressions. The constructor was changed to accept
        a \code{Map[Function, Expr]}. A \code{Function} is either a
        \code{FunctionVariable}, \code{d}, \code{dd} or
        \code{noOrder}. Where \code{noOrder} is a \code{Function} object 
        with the sole purpose of identifying that part of the pde. In
        \begin{align*}
            \alpha u_{xx}+\beta u_{tt}+\gamma u_{xt}+\zeta u_x + \nu
            u_t + \xi u + f(x, t) = 0
        \end{align*}
        $f(x, t)$ would be \code{noOrder}.

        
    \item Implemented a solver for general bivariate pdes. Only the
        elliptical case has been covered and various code surrounding
        validity intervals are still in the works. If the function
        only has valid constant coefficients then it works right now. 
    \end{itemize}
\item[July 22th:]\begin{itemize}

    \item Implemented a solver for first order PDEs. Tested with a
        fairly simple example, but it works fine. The error code still
        has to be worked in since it is just a rework of Laplace's
        error right now. 
        
    \item Continued work on the error value data type and the solution. The class
        should work such that.
\begin{verbatim}
val point56 = solution(5,6)
// point56 = 1.2 +/- 0.4
val noErrorAdd = point56 + 2.0
// 3.2 +/- 0.4
solution(-1000, -1000)
// throws out of bounds exception
\end{verbatim}
        So the solution is stored in a data type where one can
        retrieve the value of all points within bounds. This value is
        returned in a new data type (\code{errorVal}). \code{errorVal}
        interacts with \code{Doubles} by while keeping the same
        error. \code{errorVal} interacts with others of the same type
        by adding the errors appropriately.
        
    \end{itemize}

\item[July 29th]\begin{itemize}

    \item Wrote this report.
    \item Started a cleanup of the code. Lots of the ideas we're added
        on-the-fly to test an idea. Unfortunately the ideas which
        were kept were left in the code in their experimental forms. 
    \end{itemize}
    
\end{itemize}
\section{TODO}
\begin{itemize}

\item[More Testing:] Most of the examples I've tested on we're fairly
    simple ones. Some more complicated examples should probably be
    tested on. This would also include writting a propper unit testing
    section instead of just printing to the command line.

\item[\code{errorVal}:] This data type still has to be properly
    implemented. As well as the container for the solution.
    
\item[Errors:] Fix the error calculation on some of the points. 

\item[Speed Testing:] I haven't really tested much in terms of
    speed. I've messed arround with the step sizes and max iterations,
    but that was mostly so that I could get an idea of what's
    happening in a reasonable time.

\item[Comments:] Comments should probably be sprinkled here and there
    to explain some of the code. 
\end{itemize}
\end{document}
