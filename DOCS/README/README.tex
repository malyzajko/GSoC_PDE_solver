%% ---------------------------------------------------------------------------
%% --------------Jamob The Elpehant's Custom Physics Header--------------------
%% ----------------------------------------------------------------------------
%% Version 0.1

% Use %% ---------------------------------------------------------------------------
%% --------------Jamob The Elpehant's Custom Physics Header--------------------
%% ----------------------------------------------------------------------------
%% Version 0.1

% Use %% ---------------------------------------------------------------------------
%% --------------Jamob The Elpehant's Custom Physics Header--------------------
%% ----------------------------------------------------------------------------
%% Version 0.1

% Use \input{header.tex} at the top of the file to use this
% Comment in and out as you may see fit


%% --------------------General Stuff------------------------------------------
\documentclass[11pt]{article}
\usepackage{amsmath}  % AMSMath Package
\usepackage{amsthm}   % Theorem Formatting
\usepackage{amssymb}  % Math symbols like \mathbb
\usepackage{graphics} % EPS images
\usepackage{multicol} % Multiple columns in tables
\usepackage{multirow} % Multiple rows in tables
% Sets margins to something reasonable
\usepackage{cancel} % To cross out stuff
\usepackage{enumerate}
\renewcommand{\labelitemii}{(\alph{enumi})}
% Use letters for 2nd level enumeration
\usepackage{fancyhdr}
\newcommand{\header}[4]{\pagestyle{fancy}\lhead{#1\\#2}\rhead{#3\\#4}}

\usepackage{mathtools} % TO BOX ACROSS ALIGNMENTS TABS!!!111 :D

% -----------------------------General Math------------------------------------
\newcommand{\lrbrace}[1]{\left(#1\right)}
\newcommand{\lrbraces}[1]{\left[#1\right]}
% Auto-sized braces because I'm lazy
\let\oldepsilon=\epsilon
\let\oldphi=\phi
\renewcommand{\epsilon}{\varepsilon}
\renewcommand{\phi}{\varphi}
% Renaming epsilon and phi to the cool epsilon and phi
\usepackage{calligra}
\DeclareMathAlphabet{\mathcalligra}{T1}{calligra}{m}{n}
\DeclareFontShape{T1}{calligra}{m}{n}{<->s*[2.2]callig15}{}
\newcommand{\scripty}[1]{\ensuremath{\mathcalligra{#1}}}
% For scripted letters, like \scirpty{r}
\newcommand{\avg}[1]{\left\langle #1\right\rangle}
\newcommand{\abs}[1]{\left\lvert #1\right\rvert}
\newcommand{\cotan}{\mbox{cotan}}
% Because for some reason it's undefined
\newcommand{\multequal}[4]
{
    \left\{
        \begin{array}{l l}
            #1&#2\\\\
            #3&#4
        \end{array}
    \right.
}
% 1 and 3 are the values, 2 and 4 are the conditions

% ------------------------------Linear Algebra----------------------------------
\let\vaccent=\v % Renaming the funny \v{} accent to \vaccent{}
\renewcommand{\v}[1]{\ensuremath{\mathbf{#1}}}
\newcommand{\gv}[1]{\ensuremath{\mathbf{#1}}}
% Alternative representation of vectors, bold vs the arrow
\DeclareMathOperator{\Tr}{Tr}
\newcommand{\trace}[1]{\ensuremath{\Tr\left(#1\right)}}
% Two alternatives for trace, one comes with auto-sized brakets
\newcommand{\ptrace}[2]{\Tr_{#1}\left(#2\right)}
% Partial trace
\newcommand{\uv}[1]{\mathbf{\hat{#1}}} % for unit vector
\newcommand{\I}{\ensuremath{\mathbb{I}}}
\newcommand{\commute}[2]{\left[#1, #2\right]}


% ----------------------------------Quantum-------------------------------------
\newcommand{\bra}[1]{\ensuremath{\left\langle #1\right\rvert}}
\newcommand{\ket}[1]{\ensuremath{\left\lvert #1\right\rangle}}
\newcommand{\braket}[2]{\ensuremath{\left\langle #1\right\lvert\left.
            #2\right\rangle}}
\newcommand{\ketbra}[2]{\ensuremath{\left\lvert
            #1\right\rangle\left\langle #2\right\rvert}}
\newcommand{\hammy}{\ensuremath{\hat{H}}} % Hamiltonian
\newcommand{\rhohat}{\hat{\rho}}
\newcommand{\shat}[1]{\hat{S}_{#1}}
\newcommand{\lhat}[1]{hat{L}_{#1}}


% ----------------------------------Calculus------------------------------------
\newcommand{\del}{\patrial} % Because typing \patrial is long
\newcommand{\dbar}{\mathchar'26\mkern-10mud}
\let\underdot=\d
\renewcommand{\d}[2]{\frac{d #1}{d #2}}
\newcommand{\dd}[2]{\frac{d^2 #1}{d #2^2}}
\newcommand{\pd}[2]{\frac{\partial #1}{\partial #2}}
\newcommand{\pdd}[2]{\frac{\partial^2 #1}{\partial #2^2}}
\newcommand{\pdc}[3]{\frac{\partial^2 #1}{\partial #2\partial #3}}
\newcommand{\grad}[1]{\gv{\nabla} #1}
\newcommand{\curl}[1]{\gv{\nabla}\times \gv{#1}}
\newcommand{\dive}[1]{\gv{\nabla}\cdot \gv{#1}}
% I actually use the \div symbol, so I didn't name it \div
\newcommand{\indint}[4]{\displaystyle\int_{#1}^{#2}#3\ d#4}
\newcommand{\inteval}[3]{#1\bigg\rvert_{#2}^{#3}}
\newcommand{\deval}[2]{#1\bigg\rvert_{#2}}

% --------------------------Electricity and Magnetism--------------------------
\newcommand{\hfield}{\v{H}}
\newcommand{\bfield}{\v{B}}
\newcommand{\efield}{\v{E}}
\newcommand{\econstant}{\epsilon_0}
\newcommand{\mconstant}{\mu_0}

% ---------------------------Theorem Stuff--------------------------------------

\newtheorem{prop}{Proposition}
\newtheorem{thm}{Theorem}[section]
\newtheorem{lem}[thm]{Lemma}
\theoremstyle{definition}
\newtheorem{dfn}{Definition}
\theoremstyle{remark}
\newtheorem*{rmk}{Remark}

\newcounter{definition}
\newenvironment{definition}[1][]{
    \refstepcounter{definition}\par\medskip\noindent%
    \textbf{Definition~\thesection.\thedefinition.} \emph{#1} \rmfamily}{\medskip}

\newcounter{algorithm}
\newenvironment{algorithm}[1][]{
    \refstepcounter{algorithm}\par\medskip\noindent%
    \textbf{Algorithm~\thesection.\thealgorithm.} \emph{#1}
    \rmfamily}{\medskip}

\newcounter{example}
\newenvironment{example}[1][]{\refstepcounter{example}\par\medskip\noindent%
  \textbf{Example~\thesection.\theexample. #1}\rmfamily}{\medskip}

\newcounter{lemma}
\newenvironment{lemma}[1][]{\refstepcounter{lemma}\par\medskip\noindent%
  \textbf{Lemma~\thesection.\thelemma. #1}\rmfamily}{\medskip}
 at the top of the file to use this
% Comment in and out as you may see fit


%% --------------------General Stuff------------------------------------------
\documentclass[11pt]{article}
\usepackage{amsmath}  % AMSMath Package
\usepackage{amsthm}   % Theorem Formatting
\usepackage{amssymb}  % Math symbols like \mathbb
\usepackage{graphics} % EPS images
\usepackage{multicol} % Multiple columns in tables
\usepackage{multirow} % Multiple rows in tables
% Sets margins to something reasonable
\usepackage{cancel} % To cross out stuff
\usepackage{enumerate}
\renewcommand{\labelitemii}{(\alph{enumi})}
% Use letters for 2nd level enumeration
\usepackage{fancyhdr}
\newcommand{\header}[4]{\pagestyle{fancy}\lhead{#1\\#2}\rhead{#3\\#4}}

\usepackage{mathtools} % TO BOX ACROSS ALIGNMENTS TABS!!!111 :D

% -----------------------------General Math------------------------------------
\newcommand{\lrbrace}[1]{\left(#1\right)}
\newcommand{\lrbraces}[1]{\left[#1\right]}
% Auto-sized braces because I'm lazy
\let\oldepsilon=\epsilon
\let\oldphi=\phi
\renewcommand{\epsilon}{\varepsilon}
\renewcommand{\phi}{\varphi}
% Renaming epsilon and phi to the cool epsilon and phi
\usepackage{calligra}
\DeclareMathAlphabet{\mathcalligra}{T1}{calligra}{m}{n}
\DeclareFontShape{T1}{calligra}{m}{n}{<->s*[2.2]callig15}{}
\newcommand{\scripty}[1]{\ensuremath{\mathcalligra{#1}}}
% For scripted letters, like \scirpty{r}
\newcommand{\avg}[1]{\left\langle #1\right\rangle}
\newcommand{\abs}[1]{\left\lvert #1\right\rvert}
\newcommand{\cotan}{\mbox{cotan}}
% Because for some reason it's undefined
\newcommand{\multequal}[4]
{
    \left\{
        \begin{array}{l l}
            #1&#2\\\\
            #3&#4
        \end{array}
    \right.
}
% 1 and 3 are the values, 2 and 4 are the conditions

% ------------------------------Linear Algebra----------------------------------
\let\vaccent=\v % Renaming the funny \v{} accent to \vaccent{}
\renewcommand{\v}[1]{\ensuremath{\mathbf{#1}}}
\newcommand{\gv}[1]{\ensuremath{\mathbf{#1}}}
% Alternative representation of vectors, bold vs the arrow
\DeclareMathOperator{\Tr}{Tr}
\newcommand{\trace}[1]{\ensuremath{\Tr\left(#1\right)}}
% Two alternatives for trace, one comes with auto-sized brakets
\newcommand{\ptrace}[2]{\Tr_{#1}\left(#2\right)}
% Partial trace
\newcommand{\uv}[1]{\mathbf{\hat{#1}}} % for unit vector
\newcommand{\I}{\ensuremath{\mathbb{I}}}
\newcommand{\commute}[2]{\left[#1, #2\right]}


% ----------------------------------Quantum-------------------------------------
\newcommand{\bra}[1]{\ensuremath{\left\langle #1\right\rvert}}
\newcommand{\ket}[1]{\ensuremath{\left\lvert #1\right\rangle}}
\newcommand{\braket}[2]{\ensuremath{\left\langle #1\right\lvert\left.
            #2\right\rangle}}
\newcommand{\ketbra}[2]{\ensuremath{\left\lvert
            #1\right\rangle\left\langle #2\right\rvert}}
\newcommand{\hammy}{\ensuremath{\hat{H}}} % Hamiltonian
\newcommand{\rhohat}{\hat{\rho}}
\newcommand{\shat}[1]{\hat{S}_{#1}}
\newcommand{\lhat}[1]{hat{L}_{#1}}


% ----------------------------------Calculus------------------------------------
\newcommand{\del}{\patrial} % Because typing \patrial is long
\newcommand{\dbar}{\mathchar'26\mkern-10mud}
\let\underdot=\d
\renewcommand{\d}[2]{\frac{d #1}{d #2}}
\newcommand{\dd}[2]{\frac{d^2 #1}{d #2^2}}
\newcommand{\pd}[2]{\frac{\partial #1}{\partial #2}}
\newcommand{\pdd}[2]{\frac{\partial^2 #1}{\partial #2^2}}
\newcommand{\pdc}[3]{\frac{\partial^2 #1}{\partial #2\partial #3}}
\newcommand{\grad}[1]{\gv{\nabla} #1}
\newcommand{\curl}[1]{\gv{\nabla}\times \gv{#1}}
\newcommand{\dive}[1]{\gv{\nabla}\cdot \gv{#1}}
% I actually use the \div symbol, so I didn't name it \div
\newcommand{\indint}[4]{\displaystyle\int_{#1}^{#2}#3\ d#4}
\newcommand{\inteval}[3]{#1\bigg\rvert_{#2}^{#3}}
\newcommand{\deval}[2]{#1\bigg\rvert_{#2}}

% --------------------------Electricity and Magnetism--------------------------
\newcommand{\hfield}{\v{H}}
\newcommand{\bfield}{\v{B}}
\newcommand{\efield}{\v{E}}
\newcommand{\econstant}{\epsilon_0}
\newcommand{\mconstant}{\mu_0}

% ---------------------------Theorem Stuff--------------------------------------

\newtheorem{prop}{Proposition}
\newtheorem{thm}{Theorem}[section]
\newtheorem{lem}[thm]{Lemma}
\theoremstyle{definition}
\newtheorem{dfn}{Definition}
\theoremstyle{remark}
\newtheorem*{rmk}{Remark}

\newcounter{definition}
\newenvironment{definition}[1][]{
    \refstepcounter{definition}\par\medskip\noindent%
    \textbf{Definition~\thesection.\thedefinition.} \emph{#1} \rmfamily}{\medskip}

\newcounter{algorithm}
\newenvironment{algorithm}[1][]{
    \refstepcounter{algorithm}\par\medskip\noindent%
    \textbf{Algorithm~\thesection.\thealgorithm.} \emph{#1}
    \rmfamily}{\medskip}

\newcounter{example}
\newenvironment{example}[1][]{\refstepcounter{example}\par\medskip\noindent%
  \textbf{Example~\thesection.\theexample. #1}\rmfamily}{\medskip}

\newcounter{lemma}
\newenvironment{lemma}[1][]{\refstepcounter{lemma}\par\medskip\noindent%
  \textbf{Lemma~\thesection.\thelemma. #1}\rmfamily}{\medskip}
 at the top of the file to use this
% Comment in and out as you may see fit


%% --------------------General Stuff------------------------------------------
\documentclass[11pt]{article}
\usepackage{amsmath}  % AMSMath Package
\usepackage{amsthm}   % Theorem Formatting
\usepackage{amssymb}  % Math symbols like \mathbb
\usepackage{graphics} % EPS images
\usepackage{multicol} % Multiple columns in tables
\usepackage{multirow} % Multiple rows in tables
% Sets margins to something reasonable
\usepackage{cancel} % To cross out stuff
\usepackage{enumerate}
\renewcommand{\labelitemii}{(\alph{enumi})}
% Use letters for 2nd level enumeration
\usepackage{fancyhdr}
\newcommand{\header}[4]{\pagestyle{fancy}\lhead{#1\\#2}\rhead{#3\\#4}}

\usepackage{mathtools} % TO BOX ACROSS ALIGNMENTS TABS!!!111 :D

% -----------------------------General Math------------------------------------
\newcommand{\lrbrace}[1]{\left(#1\right)}
\newcommand{\lrbraces}[1]{\left[#1\right]}
% Auto-sized braces because I'm lazy
\let\oldepsilon=\epsilon
\let\oldphi=\phi
\renewcommand{\epsilon}{\varepsilon}
\renewcommand{\phi}{\varphi}
% Renaming epsilon and phi to the cool epsilon and phi
\usepackage{calligra}
\DeclareMathAlphabet{\mathcalligra}{T1}{calligra}{m}{n}
\DeclareFontShape{T1}{calligra}{m}{n}{<->s*[2.2]callig15}{}
\newcommand{\scripty}[1]{\ensuremath{\mathcalligra{#1}}}
% For scripted letters, like \scirpty{r}
\newcommand{\avg}[1]{\left\langle #1\right\rangle}
\newcommand{\abs}[1]{\left\lvert #1\right\rvert}
\newcommand{\cotan}{\mbox{cotan}}
% Because for some reason it's undefined
\newcommand{\multequal}[4]
{
    \left\{
        \begin{array}{l l}
            #1&#2\\\\
            #3&#4
        \end{array}
    \right.
}
% 1 and 3 are the values, 2 and 4 are the conditions

% ------------------------------Linear Algebra----------------------------------
\let\vaccent=\v % Renaming the funny \v{} accent to \vaccent{}
\renewcommand{\v}[1]{\ensuremath{\mathbf{#1}}}
\newcommand{\gv}[1]{\ensuremath{\mathbf{#1}}}
% Alternative representation of vectors, bold vs the arrow
\DeclareMathOperator{\Tr}{Tr}
\newcommand{\trace}[1]{\ensuremath{\Tr\left(#1\right)}}
% Two alternatives for trace, one comes with auto-sized brakets
\newcommand{\ptrace}[2]{\Tr_{#1}\left(#2\right)}
% Partial trace
\newcommand{\uv}[1]{\mathbf{\hat{#1}}} % for unit vector
\newcommand{\I}{\ensuremath{\mathbb{I}}}
\newcommand{\commute}[2]{\left[#1, #2\right]}


% ----------------------------------Quantum-------------------------------------
\newcommand{\bra}[1]{\ensuremath{\left\langle #1\right\rvert}}
\newcommand{\ket}[1]{\ensuremath{\left\lvert #1\right\rangle}}
\newcommand{\braket}[2]{\ensuremath{\left\langle #1\right\lvert\left.
            #2\right\rangle}}
\newcommand{\ketbra}[2]{\ensuremath{\left\lvert
            #1\right\rangle\left\langle #2\right\rvert}}
\newcommand{\hammy}{\ensuremath{\hat{H}}} % Hamiltonian
\newcommand{\rhohat}{\hat{\rho}}
\newcommand{\shat}[1]{\hat{S}_{#1}}
\newcommand{\lhat}[1]{hat{L}_{#1}}


% ----------------------------------Calculus------------------------------------
\newcommand{\del}{\patrial} % Because typing \patrial is long
\newcommand{\dbar}{\mathchar'26\mkern-10mud}
\let\underdot=\d
\renewcommand{\d}[2]{\frac{d #1}{d #2}}
\newcommand{\dd}[2]{\frac{d^2 #1}{d #2^2}}
\newcommand{\pd}[2]{\frac{\partial #1}{\partial #2}}
\newcommand{\pdd}[2]{\frac{\partial^2 #1}{\partial #2^2}}
\newcommand{\pdc}[3]{\frac{\partial^2 #1}{\partial #2\partial #3}}
\newcommand{\grad}[1]{\gv{\nabla} #1}
\newcommand{\curl}[1]{\gv{\nabla}\times \gv{#1}}
\newcommand{\dive}[1]{\gv{\nabla}\cdot \gv{#1}}
% I actually use the \div symbol, so I didn't name it \div
\newcommand{\indint}[4]{\displaystyle\int_{#1}^{#2}#3\ d#4}
\newcommand{\inteval}[3]{#1\bigg\rvert_{#2}^{#3}}
\newcommand{\deval}[2]{#1\bigg\rvert_{#2}}

% --------------------------Electricity and Magnetism--------------------------
\newcommand{\hfield}{\v{H}}
\newcommand{\bfield}{\v{B}}
\newcommand{\efield}{\v{E}}
\newcommand{\econstant}{\epsilon_0}
\newcommand{\mconstant}{\mu_0}

% ---------------------------Theorem Stuff--------------------------------------

\newtheorem{prop}{Proposition}
\newtheorem{thm}{Theorem}[section]
\newtheorem{lem}[thm]{Lemma}
\theoremstyle{definition}
\newtheorem{dfn}{Definition}
\theoremstyle{remark}
\newtheorem*{rmk}{Remark}

\newcounter{definition}
\newenvironment{definition}[1][]{
    \refstepcounter{definition}\par\medskip\noindent%
    \textbf{Definition~\thesection.\thedefinition.} \emph{#1} \rmfamily}{\medskip}

\newcounter{algorithm}
\newenvironment{algorithm}[1][]{
    \refstepcounter{algorithm}\par\medskip\noindent%
    \textbf{Algorithm~\thesection.\thealgorithm.} \emph{#1}
    \rmfamily}{\medskip}

\newcounter{example}
\newenvironment{example}[1][]{\refstepcounter{example}\par\medskip\noindent%
  \textbf{Example~\thesection.\theexample. #1}\rmfamily}{\medskip}

\newcounter{lemma}
\newenvironment{lemma}[1][]{\refstepcounter{lemma}\par\medskip\noindent%
  \textbf{Lemma~\thesection.\thelemma. #1}\rmfamily}{\medskip}

\usepackage{amsmath}
\usepackage{graphicx}
\usepackage{epsfig}
\usepackage{amsmath}
\usepackage{hyperref}
\usepackage{amssymb}
\usepackage{amsthm}
\usepackage{booktabs}
\usepackage{stmaryrd}
\usepackage{setspace}
\usepackage{url}
\usepackage{listings}
\newcommand{\verba}[1]{{\ttfamily #1}}
\usepackage{pgf,tikz}
\usetikzlibrary{arrows}
\usepackage{longtable}
\usepackage[figuresright]{rotating}
\usepackage{mathrsfs}
\usepackage[margin=1in]{geometry}

\title{Finite Difference Numerical PDE Solver in Scala}
\date{}

\begin{document}
\maketitle
\tableofcontents
\pagebreak

\section{Introduction}
\label{sec:Intro}
\onehalfspacing
Partial differential equations are used in most fields to
model various phenomena. In science they are used to model heat,
sound, electrostatics, electrodynamics, fluids, elasticity, bacteria
growth and so on. Outside of science, they are used in economics to
model asset pricings (see Black-Scholes model).

Despite their widespread useage, many PDEs are not simple to solve
explicitly. For those cases, one must rely numerical methods. There
exists many types of numerical methods to solve PDEs, but the three
most common ones are: the finite difference method, where one
approximates functions by values at certain grid points and
derivatives are appproximated through differences in these values; the
finite element method, where functions are split into baiss functions
and then the PDE is solved in integral form; finally, there is the
spectral method which represents functions as a sum of basis
functions.

This project implemented various forms of the finite difference method
to solve different categories of PDEs.

\section{Theory}
\subsection{Finite Differences}
When using the finite difference method, one has to assume that the
function is well behaved, thus we can expand it in a Taylor's series.
\begin{align*}
    f\lrbrace{x_0+h}&= f\lrbrace{x_0} +\frac{f'\lrbrace{x_0}}{1!}h +
    \frac{f''\lrbrace{x_0}}{2!}h^2+ ...\\
    & = f\lrbrace{x_0} +\frac{f'\lrbrace{x_0}}{1!}h + O\lrbrace{h^2}
\end{align*}
where $h$ is the step size. From there we can write out
\begin{align*}
    f'\lrbrace{x_0}& = \frac{f\lrbrace{x_0+h}-f\lrbrace{x_0}}{h} +
    \frac{1}{h}O\lrbrace{h^2}\\
    f'\lrbrace{x_0}& \approx \frac{f\lrbrace{x_0+h}-f\lrbrace{x_0}}{h}
\end{align*}
This particular approximation would be called a forwards approximation
because we take a step forwards in $x$ to obtain the derivative's
approximation. A backwards step would be
\begin{align*}
    f'\lrbrace{x_0}& \approx \frac{f\lrbrace{x_0}-f\lrbrace{x_0-h}}{h}
\end{align*}
In this case, the truncation error is $\cfrac{f''\lrbrace{x_0}}{2!}h$
which we obtain because we cut off the Taylor expansion early.

Other derivative discretizations for higher order, bivariate functions
are
\begin{align*}
    \pd{u}{x}&\approx \frac{u(x+\Delta x, t) - u(x, t)}{\Delta x}\\\\
    \pd{u}{t}&\approx \frac{u(x, t+\Delta t) - u(x, t)}{\Delta t}\\\\
    \pdd{u}{x}&\approx \frac{u(x+\Delta x, t)- 2u(x, t)+ u(x-\Delta,
        t)}{\Delta x^2}\\\\
    \pdd{u}{t}&\approx \frac{u(x, t+\Delta t)- 2u(x, t)+ u(x,
        t+\Delta t)}{\Delta t^2}\\ \\
    \frac{\partial u}{\partial x\partial t} =   \frac{\partial
        u}{\partial t\partial x}&\approx
    \frac{u(x+\Delta x, t+\Delta t)- u(x+\Delta x, t-\Delta t) -
        u(x-\Delta x, t+\Delta t) + u(x-\Delta x, t-\delta t)}{4\Delta
        x\Delta t}
\end{align*}

\subsection{Generating Solutions to PDEs}

A simple first order PDE looks like
\begin{align*}
    a\pd{u}{x}+b\pd{u}{t}+cu(x, t)+ f(x, t) = 0
\end{align*}
Where $a, b$ and $c$ are functions of $x$ and $t$. First we split up
the domain of the function into a grid. So we have $x_i$ where
$0<i<m$ where $x_0$ is the minimum value on the boundary $x$ can
have and $x_m$ is the maximum. Likewise, we have $t_j$ with $0<j<n$.
Now at these grid points we shall write $u(x_i, t_j)$ as $u_{i,
    j}$ and similarly for $a, b, c$ and $f$.
\begin{align}\label{eqn:ex1}
    a_{i, j}\lrbrace{\frac{u_{i+1, j} - u_{i-1, j}}{2h}} +  b_{i,
        j}\lrbrace{\frac{u_{i, j+1} - u_{i, j}}{k}} + c_{i, j}u_{i, j}
    + f_{i, j} = 0
\end{align}
where we have approximated the $x$ derivative by a central
difference.
Now we can explicitly solve for $u_{i, j+1}$ as
\begin{align*}
    u_{i, j+1} & = \frac{k}{b_{i, j}}\lrbrace{a_{i, j}\lrbrace{\frac{u_{i-1, j} - u_{i+1,
                    j}}{2h}} + \lrbrace{\frac{b_{i, j}}{k} - c_{i,
                j}}u_{i, j} - f_{i, j}}
\end{align*}
As shown in ~\ref{fig:stencil}.
\begin{figure}
    \begin{center}
        \definecolor{qqqqff}{rgb}{0,0,1}
        \begin{tikzpicture}[line cap=round,line join=round,>=triangle 45,x=1.0cm,y=1.0cm]
            \clip(-3.16,-0.96) rectangle (3.75,2.62);
            \draw (0,0)-- (0,2);
            \draw (-2,0)-- (2,0);
            \draw (1.87,0.35) node[anchor=north west] {$u_{i+1, j}$};
            \draw (-2.18,0.35) node[anchor=north west] {$u_{i-1, j}$};
            \draw (0.09,0.33) node[anchor=north west] {$u_{i, j}$};
            \draw (0.09,2.19) node[anchor=north west] {$u_{i, j+1}$};
            \begin{scriptsize}
                \fill [color=qqqqff] (0,0) circle (1.5pt);
                \fill [color=qqqqff] (0,2) circle (1.5pt);
                \fill [color=qqqqff] (-2,0) circle (1.5pt);
                \fill [color=qqqqff] (2,0) circle (1.5pt);
            \end{scriptsize}
        \end{tikzpicture}
    \end{center}
    \caption{Explicit method for first order PDEs.}
    \label{fig:stencil}
\end{figure}
\medskip

Many other methods exist to solve PDEs, but they all involve similar
ideas. That is, discretize the derivatives in the PDE so that one may
approximate it by nearby grid-points. For further reading refer to Finite Difference Methods for
Differential Equations by Randall J. LeVeque\cite{leveque}.
\section{Examples}
\subsection{Useage}
To use the solver, first define the variables of the function
\begin{verbatim}
import pde.model.expression._
val x = Variable("x")
val t = Variable("t")
val u = FuntionVariable("u", x, t)
\end{verbatim}
Then define the pde and boundary conditions
\begin{verbatim}
val pde = d(u, t) := 4*dd(u, x, x)
val boundary = Boundary(
(u(x, 0) := Sin(Pi*x/L), from(0 to 10)),
(u(0, t) := 0, from(0 to 2*L)),
(u(L, t) := 0, from(0 to 2*L))
)
\end{verbatim}
and finally solve it using
\begin{verbatim}
val solution = Solver.solve(pde, boundary)
val point11 = solution(1, 1)
\end{verbatim}
optionally giving arguments for $xstep$ and $tstep$.
\subsection{First Order PDEs}
These two PDEs are simple polynomials.

\noindent Sample Code
\begin{verbatim}
object FirstOrderTest extends App {
  val x = Variable("x")
  val t = Variable("t")
  val u = new FunctionVariable("u", x, t)
  val testPDE = t*d(u, t) + x*d(u, x)  - u := 0
  val boundary = Boundary(
    (u(x, 0) := 2*x, from(0 to 10)),
    (u(0, t) := 3*t, from(0 to 10)),
    (u(10, t):= 20+3*t, from(0 to 10))
  )
  val boundaryR = Boundary(
    (u(x, 0) := 2*x, from(0 to 10)),
    (u(0, t) := 3*t, from(0 to 10)),
    (u(10, t):= 20+3*t, from(0 to 10)),
    (u(x, 10):= 2*x+30, from(0 to 10))
  )
  val testPDE2 = d(u, t) + d(u, x) - u := -x*t
  val boundary2 = Boundary(
    (u(x, 0) := x+2, from(0 to 10)),
    (u(0, t) := t+2, from(0 to 10)),
    (u(10, t):= 10+2+t+10*t, from(0 to 10)),
    (u(x, 10) := 10+2+x+4*x, from(0 to 10))
  )
  def time[A](f: => A) = {
    val s = System.nanoTime
    val ret = f
    println("time: "+(System.nanoTime-s)/1e6+"ms")
    ret
  }

  val realSolution = (x: Double, t: Double) => 2*x+3*t;
  val realSolution2 = (x: Double, t: Double) => x*t+x+t+2;
  val solution1 = {
    time {Solver.solve(testPDE, boundary, xstep = 0.1, tstep = 0.1)}
  }

  val solution2 = {
    time {Solver.solve(testPDE2, boundary2, xstep = 0.1, tstep = 0.1)}
  }

}
\end{verbatim}
\emph{For brevity, the print and import statements were ommitted.}

\noindent The output:
\begingroup
\fontsize{10pt}{12pt}
\begin{verbatim}
 1.0 * u_t + 1.0 * u_x + (-(1.0)) * u + -((-(x) * t)) = 0
 time: 89664.145544ms
 Solution to (t) * u_t + (x) * u_x + (-(1.0)) * u = 0
 Generated Point (0.1, 0.1): [0.4499999999999983,0.5500000000000022]  (1.453004383478182E-15)(abs)
 0.5
 Real: 0.5
 Generated Point (0.5, 0.5): [2.449999999999945,2.550000000000058]  (4.4134289052031474E-14)(abs)
 2.5000000000000004
 Real: 2.5
 Generated Point (1, 1): [4.949999999999765,5.050000000000244]  (1.8882525167485055E-13)(abs)
 5.000000000000001
 Real: 5.0
 Generated Point(5, 5): [24.94999999999385,25.05000000000611]  (4.7692331633346055E-12)(abs)
 25.000000000000004
 Real: 25.0
 Generated Point(7.5, 7.5): [37.44999999998569,37.55000000001416]  (1.1210878933689465E-11)(abs)
 37.500000000000014
 Real: 37.5
 Generated Point(9.9, 9.9): [47.01999999997645,51.98000000002320]  (1.8086037561786404E-11)(abs)
 49.500000000000014
 Real: 49.5
 ---------------------------------------------------------------------------------------
 time: 17245.412616ms
 Solution to 1.0 * u_t + 1.0 * u_x + (-(1.0)) * u + -((-(x) * t)) = 0
 Generated Point (0.1, 0.1): [2.187999999999995,2.232000000000006]  (4.355657724309832E-15)(abs)
 2.21
 Real: 2.21
 Generated Point (0.5, 0.5): [3.219999999999946,3.280000000000056]  (4.535128333101159E-14)(abs)
 3.25
 Real: 3.25
 Generated Point (1, 1): [4.959999999999803,5.040000000000203]  (1.6680218275411806E-13)(abs)
 4.999999999999997
 Real: 5.0
 Generated Point(5, 5): [36.87999999996662,37.12000000003413]  (2.810412682693202E-11)(abs)
 36.999999999999304
 Real: 37.0
 Generated Point(7.5, 7.5): [73.07999999956860,73.42000000044115]  (3.621061309333247E-10)(abs)
 73.24999999999027
 Real: 73.25
 Generated Point(9.9, 9.9): [116.8399999951858,122.7800000049254]  (4.024409208844229E-9)(abs)
 119.80999999988663
 Real: 119.81000000000002
\end{verbatim}
\endgroup
Where the first line, the one with ``Generated Point'' is the point in
interval notation, and the line right under it is the central
value. The third line, ``Real'', is the exact value.
\subsection{Laplace's Equation}
Code:
\begin{verbatim}
object SecondOrderTest extends App {
  val x = new Variable("x")
  val y = Variable("y")
  val u = new FunctionVariable("u", x, y)
  val laplace = dd(u, x, x) + dd(u, y, y) := Const(0)
  val boundary = new RectBoundary(
    (u(x, 0) := 100, from(0 to 10)),
    (u(x, 40):= 0, from(0 to 10)),
    (u(0, y):=0, from(0 to 40)),
    (u(10, y):=0, from(0 to 40))
  )
  val solution= Solver.solve(laplace, boundary, xstep = 0.2)

  val realLSolution = (x: Double, t: Double) =>
    (400/Pi)*(exp(-Pi*y/10)*sin(Pi*x/10) +
    (1/3)*exp(-Pi*3*y/10)*sin(Pi*3*x/10) + (1/5)*exp(-Pi*y/2)*sin(Pi*x/2)
    + (1/7)*exp(-Pi*7*y/10)*sin(Pi*7*x/10) + (1/9)*exp(-Pi*9*y/10)*sin(Pi*9*x/10))
}
\end{verbatim}
Output
\begingroup
\fontsize{10pt}{12pt}
\begin{verbatim}
 Running SecondOrderTest
 Generated Point (4, 2): [0.4053465601093503,0.4101618942637927]  (4.201161737557659E-14)(abs)
 0.4077542271865715
 Real: 0.40783644615089115
 Generated Point(18, 9): [1.159954366264431,1.165624827362555]  (3.525639168648441E-14)(abs)
 1.1627895968134934
 Real: 1.2984714123775343
 Generated Point(10, 5): [1.851338280392296,1.861537303840700]  (1.543027440455468E-13)(abs)
 1.8564377921164983
 Real: 1.887349271785328
\end{verbatim}
\endgroup
\section{Implementaiton}
\subsection{Modeling PDEs}
The domain specific language to represent PDEs has \verba{Expression}
as the super-class for everything. In \verba{Expression}, there are
methods defined for $+, -, *$ and $/$ when interacting with another
\verba{Expression}.  These output a new node, \verba{Add, Sub, Mul,
    Div}, according ot the operation. One type of \verba{Expression}
which is treated differently is \verba{Function}'s, which can
be either \verba{Function Variables, d} or \verba{dd}. \verba{d} and
\verba{dd} being there to represent derivatives.

For \verba{Expression}'s there is a method \verba{:=} to convert split
up the expression into functions which multiply the derivative and
then with those it outputs a PDE.

\subsection{Tracking Errors}
The truncation of floating-point values and finite precision errors on the calculations are all
taken care off by Eva Darulova's \verba{SmartFloat}
class\cite{darulova}. As for the errors related to the truncation of
the Taylor expansion, that is done separately and the error is added
after.

In all cases, when using the finite difference method, the truncation
error on a point is related to a higher order derivative. That
derivative can also be discretized. From there using the already
computed solutions points, we add the error to the solution. Going
back to the example in the introduction ~\ref{eqn:ex1}.

\begin{align*}
    a_{i, j}\lrbrace{\frac{u_{i+1, j} - u_{i-1, j}}{2h}} +  b_{i,
        j}\lrbrace{\frac{u_{i, j+1} - u_{i, j}}{k}} + c_{i, j}u_{i, j}
    + f_{i, j} = 0
\end{align*}
The term
\begin{align*}
    a_{i, j}\lrbrace{\frac{u_{i+1, j} - u_{i-1, j}}{2h}}
\end{align*}
has a main truncation error of $\cfrac{1}{6}h^2u'''(x)$, which we
approximate by
\begin{align*}
    u_{xxx}(x, t)\approx\frac{u_{i+2, j} - 2u_{i+1, j}+2u_{i-1, j}- u_{i-2, j}}{h^3}
\end{align*}
similarly for the second term we compute the error related to the
truncation and add it to the error on the solution.

One thing to note is that the truncation error on the truncation error
is not computed. For the third derivative computed above, it has error
$O\lrbrace{h^4}$ which is simply ignored. This is because, for one, we
could keep computing errors to no end, but mostly because it becomes
very insignificant. The error on $u_x$ was $O\lrbrace{h^2}$, so if we
multiply that by the error on $u_{xxx}$, we get an $O\lrbrace{h^6}$
error which is next to nothing.

The Taylor series truncation errors are all computed once all solution
points have been generated. For one, the error depends on the
surrounding points, so we need their values to compute it, but the
main reason is to avoid having the error values propagate along the
computations prematurely.

The last place where an additional piece of error may surface is when
accessing the data. Since the solution was only computed at specific
gridpoints, if the code requests a data point off the grid, yet still
inside the defined boundary, a weighted average of the surrounding
points is computed and the appropriate error is added to it.

\begin{align*}
    \bar{x} & = \frac{\sum_{i=1}^n(w_ix_i)}{\sum_{i=1}^n w_i}&
    \sigma^2_{\bar{x}}& = \frac{1}{\sum_{i=1}^n w_i}
\end{align*}

In this case the weights are the inverse distance from the closest
grid-point. The closer the grid-point is to the desired value, the
heavier it weights in the average.
\section{Experimental Results}
As far as the results go, for the examples that have been tested it
seems to work well, but there is admitedly many more PDEs it could be
tested on. In the examples above, we can see that for first order
PDEs, the results are within the error interval of the analytic
solution.

For
the Laplace's equation and other second order equations, the
truncation error may be underestimated, since the error computed is
more of an order of magnitude error than an absolute one for some of
the points nearer to the edge of the grid. The reason for that is
because the discretization of the fourth derivative involves points
plus or minus 3 grid points away. These terms are ignored near the
ledge and the formula to approximate the derivative is adjusted to
have proper magnitude.

\subsection{Speed}
Various tests we're done regarding how big the grid can be before it
becomes unfeasible to use the solver. Without increasing Java's
default heap space, the maximum size grid possible is in the
whereabouts of 1250 by 1250.

In terms of speed, there are equivalent solvers for first order
solutions using \verba{Double}s instead of \verba{SmartFloat}s. The
speed increase when computing using normal floating-points is about 10
times faster.

\section{Further Work}
Due to time restraints, not all PDEs can actually be solved using this
solver. The ones which do work are: first order equations, Laplace's
equation, Poisson's equation and the heat equation whith scalars. The
work is there for general heat equations, but the computation time was
horrendous. For the heat equation, one must solve an $n\times n$
tridiagonal matrix at each time step. Though this can be done in
linear time, when the matrix is of size $1000\times1000$ it is
incredibly slow. With scalars, the matrix does not have to be
regenerated at each time step, only the result column has to be
computed. Other specific 2nd order equations have yet to be
implemented. The reason for implementing specific solver functions is
for optimization. Solving general equations all the time is
computationally inefficient.

For general 2nd order equations, one would have to solve a slightly more
general matrix, which has not yet been implemented yet. In addition to
this, other things which would have to be implemented are solving
using Newman boundary conditions, where one specifies $\pd{u}{x}$ at
some points. In terms of boundaries, more general boundaries may also
want to be specified by the user. As it is now, the boundaries are
very rigid in their definition.
Spherical coordinates could also be usefull for radial
Laplace's equation.

\begin{thebibliography}{9}

\bibitem{leveque}
    Randal J. Leveque,
    \emph{Finite Difference Methods for Ordinary and Partial Differential Equations,
        Steady State and Time Dependent Problems}
    Society for Industrial and Applied Mathematics (SIAM),
    Philadelphia,
    July 2007.

\bibitem{darulova}
    Eva Darulova, Viktor Kuncak,
    \emph{Trustworthy Numerical Computation in Scala},
    School of Computer and Communication Sciences (I\&C) -
    Swiss Federal Institute of Technology (EPFL), Switzerland,
    2001.

\end{thebibliography}
\end{document}
