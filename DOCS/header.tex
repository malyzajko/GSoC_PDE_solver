%% ----------------------------------------------------------------------------
%% --------------Jamob The Elpehant's Custom Physics Header--------------------
%% ----------------------------------------------------------------------------
%% Version 0.1

% Use %% ---------------------------------------------------------------------------
%% --------------Jamob The Elpehant's Custom Physics Header--------------------
%% ----------------------------------------------------------------------------
%% Version 0.1

% Use %% ---------------------------------------------------------------------------
%% --------------Jamob The Elpehant's Custom Physics Header--------------------
%% ----------------------------------------------------------------------------
%% Version 0.1

% Use %% ---------------------------------------------------------------------------
%% --------------Jamob The Elpehant's Custom Physics Header--------------------
%% ----------------------------------------------------------------------------
%% Version 0.1

% Use \input{header.tex} at the top of the file to use this
% Comment in and out as you may see fit


%% --------------------General Stuff------------------------------------------
\documentclass[11pt]{article}
\usepackage{amsmath}  % AMSMath Package
\usepackage{amsthm}   % Theorem Formatting
\usepackage{amssymb}  % Math symbols like \mathbb
\usepackage{graphics} % EPS images
\usepackage{multicol} % Multiple columns in tables
\usepackage{multirow} % Multiple rows in tables
% Sets margins to something reasonable
\usepackage{cancel} % To cross out stuff
\usepackage{enumerate}
\renewcommand{\labelitemii}{(\alph{enumi})}
% Use letters for 2nd level enumeration
\usepackage{fancyhdr}
\newcommand{\header}[4]{\pagestyle{fancy}\lhead{#1\\#2}\rhead{#3\\#4}}

\usepackage{mathtools} % TO BOX ACROSS ALIGNMENTS TABS!!!111 :D

% -----------------------------General Math------------------------------------
\newcommand{\lrbrace}[1]{\left(#1\right)}
\newcommand{\lrbraces}[1]{\left[#1\right]}
% Auto-sized braces because I'm lazy
\let\oldepsilon=\epsilon
\let\oldphi=\phi
\renewcommand{\epsilon}{\varepsilon}
\renewcommand{\phi}{\varphi}
% Renaming epsilon and phi to the cool epsilon and phi
\usepackage{calligra}
\DeclareMathAlphabet{\mathcalligra}{T1}{calligra}{m}{n}
\DeclareFontShape{T1}{calligra}{m}{n}{<->s*[2.2]callig15}{}
\newcommand{\scripty}[1]{\ensuremath{\mathcalligra{#1}}}
% For scripted letters, like \scirpty{r}
\newcommand{\avg}[1]{\left\langle #1\right\rangle}
\newcommand{\abs}[1]{\left\lvert #1\right\rvert}
\newcommand{\cotan}{\mbox{cotan}}
% Because for some reason it's undefined
\newcommand{\multequal}[4]
{
    \left\{
        \begin{array}{l l}
            #1&#2\\\\
            #3&#4
        \end{array}
    \right.
}
% 1 and 3 are the values, 2 and 4 are the conditions

% ------------------------------Linear Algebra----------------------------------
\let\vaccent=\v % Renaming the funny \v{} accent to \vaccent{}
\renewcommand{\v}[1]{\ensuremath{\mathbf{#1}}}
\newcommand{\gv}[1]{\ensuremath{\mathbf{#1}}}
% Alternative representation of vectors, bold vs the arrow
\DeclareMathOperator{\Tr}{Tr}
\newcommand{\trace}[1]{\ensuremath{\Tr\left(#1\right)}}
% Two alternatives for trace, one comes with auto-sized brakets
\newcommand{\ptrace}[2]{\Tr_{#1}\left(#2\right)}
% Partial trace
\newcommand{\uv}[1]{\mathbf{\hat{#1}}} % for unit vector
\newcommand{\I}{\ensuremath{\mathbb{I}}}
\newcommand{\commute}[2]{\left[#1, #2\right]}


% ----------------------------------Quantum-------------------------------------
\newcommand{\bra}[1]{\ensuremath{\left\langle #1\right\rvert}}
\newcommand{\ket}[1]{\ensuremath{\left\lvert #1\right\rangle}}
\newcommand{\braket}[2]{\ensuremath{\left\langle #1\right\lvert\left.
            #2\right\rangle}}
\newcommand{\ketbra}[2]{\ensuremath{\left\lvert
            #1\right\rangle\left\langle #2\right\rvert}}
\newcommand{\hammy}{\ensuremath{\hat{H}}} % Hamiltonian
\newcommand{\rhohat}{\hat{\rho}}
\newcommand{\shat}[1]{\hat{S}_{#1}}
\newcommand{\lhat}[1]{hat{L}_{#1}}


% ----------------------------------Calculus------------------------------------
\newcommand{\del}{\patrial} % Because typing \patrial is long
\newcommand{\dbar}{\mathchar'26\mkern-10mud}
\let\underdot=\d
\renewcommand{\d}[2]{\frac{d #1}{d #2}}
\newcommand{\dd}[2]{\frac{d^2 #1}{d #2^2}}
\newcommand{\pd}[2]{\frac{\partial #1}{\partial #2}}
\newcommand{\pdd}[2]{\frac{\partial^2 #1}{\partial #2^2}}
\newcommand{\pdc}[3]{\frac{\partial^2 #1}{\partial #2\partial #3}}
\newcommand{\grad}[1]{\gv{\nabla} #1}
\newcommand{\curl}[1]{\gv{\nabla}\times \gv{#1}}
\newcommand{\dive}[1]{\gv{\nabla}\cdot \gv{#1}}
% I actually use the \div symbol, so I didn't name it \div
\newcommand{\indint}[4]{\displaystyle\int_{#1}^{#2}#3\ d#4}
\newcommand{\inteval}[3]{#1\bigg\rvert_{#2}^{#3}}
\newcommand{\deval}[2]{#1\bigg\rvert_{#2}}

% --------------------------Electricity and Magnetism--------------------------
\newcommand{\hfield}{\v{H}}
\newcommand{\bfield}{\v{B}}
\newcommand{\efield}{\v{E}}
\newcommand{\econstant}{\epsilon_0}
\newcommand{\mconstant}{\mu_0}

% ---------------------------Theorem Stuff--------------------------------------

\newtheorem{prop}{Proposition}
\newtheorem{thm}{Theorem}[section]
\newtheorem{lem}[thm]{Lemma}
\theoremstyle{definition}
\newtheorem{dfn}{Definition}
\theoremstyle{remark}
\newtheorem*{rmk}{Remark}

\newcounter{definition}
\newenvironment{definition}[1][]{
    \refstepcounter{definition}\par\medskip\noindent%
    \textbf{Definition~\thesection.\thedefinition.} \emph{#1} \rmfamily}{\medskip}

\newcounter{algorithm}
\newenvironment{algorithm}[1][]{
    \refstepcounter{algorithm}\par\medskip\noindent%
    \textbf{Algorithm~\thesection.\thealgorithm.} \emph{#1}
    \rmfamily}{\medskip}

\newcounter{example}
\newenvironment{example}[1][]{\refstepcounter{example}\par\medskip\noindent%
  \textbf{Example~\thesection.\theexample. #1}\rmfamily}{\medskip}

\newcounter{lemma}
\newenvironment{lemma}[1][]{\refstepcounter{lemma}\par\medskip\noindent%
  \textbf{Lemma~\thesection.\thelemma. #1}\rmfamily}{\medskip}
 at the top of the file to use this
% Comment in and out as you may see fit


%% --------------------General Stuff------------------------------------------
\documentclass[11pt]{article}
\usepackage{amsmath}  % AMSMath Package
\usepackage{amsthm}   % Theorem Formatting
\usepackage{amssymb}  % Math symbols like \mathbb
\usepackage{graphics} % EPS images
\usepackage{multicol} % Multiple columns in tables
\usepackage{multirow} % Multiple rows in tables
% Sets margins to something reasonable
\usepackage{cancel} % To cross out stuff
\usepackage{enumerate}
\renewcommand{\labelitemii}{(\alph{enumi})}
% Use letters for 2nd level enumeration
\usepackage{fancyhdr}
\newcommand{\header}[4]{\pagestyle{fancy}\lhead{#1\\#2}\rhead{#3\\#4}}

\usepackage{mathtools} % TO BOX ACROSS ALIGNMENTS TABS!!!111 :D

% -----------------------------General Math------------------------------------
\newcommand{\lrbrace}[1]{\left(#1\right)}
\newcommand{\lrbraces}[1]{\left[#1\right]}
% Auto-sized braces because I'm lazy
\let\oldepsilon=\epsilon
\let\oldphi=\phi
\renewcommand{\epsilon}{\varepsilon}
\renewcommand{\phi}{\varphi}
% Renaming epsilon and phi to the cool epsilon and phi
\usepackage{calligra}
\DeclareMathAlphabet{\mathcalligra}{T1}{calligra}{m}{n}
\DeclareFontShape{T1}{calligra}{m}{n}{<->s*[2.2]callig15}{}
\newcommand{\scripty}[1]{\ensuremath{\mathcalligra{#1}}}
% For scripted letters, like \scirpty{r}
\newcommand{\avg}[1]{\left\langle #1\right\rangle}
\newcommand{\abs}[1]{\left\lvert #1\right\rvert}
\newcommand{\cotan}{\mbox{cotan}}
% Because for some reason it's undefined
\newcommand{\multequal}[4]
{
    \left\{
        \begin{array}{l l}
            #1&#2\\\\
            #3&#4
        \end{array}
    \right.
}
% 1 and 3 are the values, 2 and 4 are the conditions

% ------------------------------Linear Algebra----------------------------------
\let\vaccent=\v % Renaming the funny \v{} accent to \vaccent{}
\renewcommand{\v}[1]{\ensuremath{\mathbf{#1}}}
\newcommand{\gv}[1]{\ensuremath{\mathbf{#1}}}
% Alternative representation of vectors, bold vs the arrow
\DeclareMathOperator{\Tr}{Tr}
\newcommand{\trace}[1]{\ensuremath{\Tr\left(#1\right)}}
% Two alternatives for trace, one comes with auto-sized brakets
\newcommand{\ptrace}[2]{\Tr_{#1}\left(#2\right)}
% Partial trace
\newcommand{\uv}[1]{\mathbf{\hat{#1}}} % for unit vector
\newcommand{\I}{\ensuremath{\mathbb{I}}}
\newcommand{\commute}[2]{\left[#1, #2\right]}


% ----------------------------------Quantum-------------------------------------
\newcommand{\bra}[1]{\ensuremath{\left\langle #1\right\rvert}}
\newcommand{\ket}[1]{\ensuremath{\left\lvert #1\right\rangle}}
\newcommand{\braket}[2]{\ensuremath{\left\langle #1\right\lvert\left.
            #2\right\rangle}}
\newcommand{\ketbra}[2]{\ensuremath{\left\lvert
            #1\right\rangle\left\langle #2\right\rvert}}
\newcommand{\hammy}{\ensuremath{\hat{H}}} % Hamiltonian
\newcommand{\rhohat}{\hat{\rho}}
\newcommand{\shat}[1]{\hat{S}_{#1}}
\newcommand{\lhat}[1]{hat{L}_{#1}}


% ----------------------------------Calculus------------------------------------
\newcommand{\del}{\patrial} % Because typing \patrial is long
\newcommand{\dbar}{\mathchar'26\mkern-10mud}
\let\underdot=\d
\renewcommand{\d}[2]{\frac{d #1}{d #2}}
\newcommand{\dd}[2]{\frac{d^2 #1}{d #2^2}}
\newcommand{\pd}[2]{\frac{\partial #1}{\partial #2}}
\newcommand{\pdd}[2]{\frac{\partial^2 #1}{\partial #2^2}}
\newcommand{\pdc}[3]{\frac{\partial^2 #1}{\partial #2\partial #3}}
\newcommand{\grad}[1]{\gv{\nabla} #1}
\newcommand{\curl}[1]{\gv{\nabla}\times \gv{#1}}
\newcommand{\dive}[1]{\gv{\nabla}\cdot \gv{#1}}
% I actually use the \div symbol, so I didn't name it \div
\newcommand{\indint}[4]{\displaystyle\int_{#1}^{#2}#3\ d#4}
\newcommand{\inteval}[3]{#1\bigg\rvert_{#2}^{#3}}
\newcommand{\deval}[2]{#1\bigg\rvert_{#2}}

% --------------------------Electricity and Magnetism--------------------------
\newcommand{\hfield}{\v{H}}
\newcommand{\bfield}{\v{B}}
\newcommand{\efield}{\v{E}}
\newcommand{\econstant}{\epsilon_0}
\newcommand{\mconstant}{\mu_0}

% ---------------------------Theorem Stuff--------------------------------------

\newtheorem{prop}{Proposition}
\newtheorem{thm}{Theorem}[section]
\newtheorem{lem}[thm]{Lemma}
\theoremstyle{definition}
\newtheorem{dfn}{Definition}
\theoremstyle{remark}
\newtheorem*{rmk}{Remark}

\newcounter{definition}
\newenvironment{definition}[1][]{
    \refstepcounter{definition}\par\medskip\noindent%
    \textbf{Definition~\thesection.\thedefinition.} \emph{#1} \rmfamily}{\medskip}

\newcounter{algorithm}
\newenvironment{algorithm}[1][]{
    \refstepcounter{algorithm}\par\medskip\noindent%
    \textbf{Algorithm~\thesection.\thealgorithm.} \emph{#1}
    \rmfamily}{\medskip}

\newcounter{example}
\newenvironment{example}[1][]{\refstepcounter{example}\par\medskip\noindent%
  \textbf{Example~\thesection.\theexample. #1}\rmfamily}{\medskip}

\newcounter{lemma}
\newenvironment{lemma}[1][]{\refstepcounter{lemma}\par\medskip\noindent%
  \textbf{Lemma~\thesection.\thelemma. #1}\rmfamily}{\medskip}
 at the top of the file to use this
% Comment in and out as you may see fit


%% --------------------General Stuff------------------------------------------
\documentclass[11pt]{article}
\usepackage{amsmath}  % AMSMath Package
\usepackage{amsthm}   % Theorem Formatting
\usepackage{amssymb}  % Math symbols like \mathbb
\usepackage{graphics} % EPS images
\usepackage{multicol} % Multiple columns in tables
\usepackage{multirow} % Multiple rows in tables
% Sets margins to something reasonable
\usepackage{cancel} % To cross out stuff
\usepackage{enumerate}
\renewcommand{\labelitemii}{(\alph{enumi})}
% Use letters for 2nd level enumeration
\usepackage{fancyhdr}
\newcommand{\header}[4]{\pagestyle{fancy}\lhead{#1\\#2}\rhead{#3\\#4}}

\usepackage{mathtools} % TO BOX ACROSS ALIGNMENTS TABS!!!111 :D

% -----------------------------General Math------------------------------------
\newcommand{\lrbrace}[1]{\left(#1\right)}
\newcommand{\lrbraces}[1]{\left[#1\right]}
% Auto-sized braces because I'm lazy
\let\oldepsilon=\epsilon
\let\oldphi=\phi
\renewcommand{\epsilon}{\varepsilon}
\renewcommand{\phi}{\varphi}
% Renaming epsilon and phi to the cool epsilon and phi
\usepackage{calligra}
\DeclareMathAlphabet{\mathcalligra}{T1}{calligra}{m}{n}
\DeclareFontShape{T1}{calligra}{m}{n}{<->s*[2.2]callig15}{}
\newcommand{\scripty}[1]{\ensuremath{\mathcalligra{#1}}}
% For scripted letters, like \scirpty{r}
\newcommand{\avg}[1]{\left\langle #1\right\rangle}
\newcommand{\abs}[1]{\left\lvert #1\right\rvert}
\newcommand{\cotan}{\mbox{cotan}}
% Because for some reason it's undefined
\newcommand{\multequal}[4]
{
    \left\{
        \begin{array}{l l}
            #1&#2\\\\
            #3&#4
        \end{array}
    \right.
}
% 1 and 3 are the values, 2 and 4 are the conditions

% ------------------------------Linear Algebra----------------------------------
\let\vaccent=\v % Renaming the funny \v{} accent to \vaccent{}
\renewcommand{\v}[1]{\ensuremath{\mathbf{#1}}}
\newcommand{\gv}[1]{\ensuremath{\mathbf{#1}}}
% Alternative representation of vectors, bold vs the arrow
\DeclareMathOperator{\Tr}{Tr}
\newcommand{\trace}[1]{\ensuremath{\Tr\left(#1\right)}}
% Two alternatives for trace, one comes with auto-sized brakets
\newcommand{\ptrace}[2]{\Tr_{#1}\left(#2\right)}
% Partial trace
\newcommand{\uv}[1]{\mathbf{\hat{#1}}} % for unit vector
\newcommand{\I}{\ensuremath{\mathbb{I}}}
\newcommand{\commute}[2]{\left[#1, #2\right]}


% ----------------------------------Quantum-------------------------------------
\newcommand{\bra}[1]{\ensuremath{\left\langle #1\right\rvert}}
\newcommand{\ket}[1]{\ensuremath{\left\lvert #1\right\rangle}}
\newcommand{\braket}[2]{\ensuremath{\left\langle #1\right\lvert\left.
            #2\right\rangle}}
\newcommand{\ketbra}[2]{\ensuremath{\left\lvert
            #1\right\rangle\left\langle #2\right\rvert}}
\newcommand{\hammy}{\ensuremath{\hat{H}}} % Hamiltonian
\newcommand{\rhohat}{\hat{\rho}}
\newcommand{\shat}[1]{\hat{S}_{#1}}
\newcommand{\lhat}[1]{hat{L}_{#1}}


% ----------------------------------Calculus------------------------------------
\newcommand{\del}{\patrial} % Because typing \patrial is long
\newcommand{\dbar}{\mathchar'26\mkern-10mud}
\let\underdot=\d
\renewcommand{\d}[2]{\frac{d #1}{d #2}}
\newcommand{\dd}[2]{\frac{d^2 #1}{d #2^2}}
\newcommand{\pd}[2]{\frac{\partial #1}{\partial #2}}
\newcommand{\pdd}[2]{\frac{\partial^2 #1}{\partial #2^2}}
\newcommand{\pdc}[3]{\frac{\partial^2 #1}{\partial #2\partial #3}}
\newcommand{\grad}[1]{\gv{\nabla} #1}
\newcommand{\curl}[1]{\gv{\nabla}\times \gv{#1}}
\newcommand{\dive}[1]{\gv{\nabla}\cdot \gv{#1}}
% I actually use the \div symbol, so I didn't name it \div
\newcommand{\indint}[4]{\displaystyle\int_{#1}^{#2}#3\ d#4}
\newcommand{\inteval}[3]{#1\bigg\rvert_{#2}^{#3}}
\newcommand{\deval}[2]{#1\bigg\rvert_{#2}}

% --------------------------Electricity and Magnetism--------------------------
\newcommand{\hfield}{\v{H}}
\newcommand{\bfield}{\v{B}}
\newcommand{\efield}{\v{E}}
\newcommand{\econstant}{\epsilon_0}
\newcommand{\mconstant}{\mu_0}

% ---------------------------Theorem Stuff--------------------------------------

\newtheorem{prop}{Proposition}
\newtheorem{thm}{Theorem}[section]
\newtheorem{lem}[thm]{Lemma}
\theoremstyle{definition}
\newtheorem{dfn}{Definition}
\theoremstyle{remark}
\newtheorem*{rmk}{Remark}

\newcounter{definition}
\newenvironment{definition}[1][]{
    \refstepcounter{definition}\par\medskip\noindent%
    \textbf{Definition~\thesection.\thedefinition.} \emph{#1} \rmfamily}{\medskip}

\newcounter{algorithm}
\newenvironment{algorithm}[1][]{
    \refstepcounter{algorithm}\par\medskip\noindent%
    \textbf{Algorithm~\thesection.\thealgorithm.} \emph{#1}
    \rmfamily}{\medskip}

\newcounter{example}
\newenvironment{example}[1][]{\refstepcounter{example}\par\medskip\noindent%
  \textbf{Example~\thesection.\theexample. #1}\rmfamily}{\medskip}

\newcounter{lemma}
\newenvironment{lemma}[1][]{\refstepcounter{lemma}\par\medskip\noindent%
  \textbf{Lemma~\thesection.\thelemma. #1}\rmfamily}{\medskip}
 at the top of the file to use this
% Comment in and out as you may see fit


%% --------------------General Stuff------------------------------------------ 
\documentclass[11pt]{article}
\usepackage{amsmath}  % AMSMath Package
\usepackage{amsthm}   % Theorem Formatting
\usepackage{amssymb}  % Math symbols like \mathbb
\usepackage{graphics} % EPS images
\usepackage{multicol} % Multiple columns in tables
\usepackage{multirow} % Multiple rows in tables
\usepackage[dvips, letterpaper, margin=0.75in, bottom=0.5in]{geometry}
% Sets margins to something reasonable
\usepackage{cancel} % To cross out stuff
\usepackage{enumerate}
\renewcommand{\labelitemii}{(\alph{enumi})}
% Use letters for 2nd level enumeration
\usepackage{fancyhdr}
\newcommand{\header}[4]{\pagestyle{fancy}\lhead{#1\\#2}\rhead{#3\\#4}}

\usepackage{mathtools} % TO BOX ACROSS ALIGNMENTS TABS!!!111 :D

% -----------------------------General Math------------------------------------
\newcommand{\lrbrace}[1]{\left(#1\right)}
\newcommand{\lrbraces}[1]{\left[#1\right]}
% Auto-sized braces because I'm lazy
\let\oldepsilon=\epsilon
\let\oldphi=\phi
\renewcommand{\epsilon}{\varepsilon}
\renewcommand{\phi}{\varphi}
% Renaming epsilon and phi to the cool epsilon and phi
\usepackage{calligra}
\DeclareMathAlphabet{\mathcalligra}{T1}{calligra}{m}{n}
\DeclareFontShape{T1}{calligra}{m}{n}{<->s*[2.2]callig15}{}
\newcommand{\scripty}[1]{\ensuremath{\mathcalligra{#1}}}
% For scripted letters, like \scirpty{r}
\newcommand{\avg}[1]{\left\langle #1\right\rangle}
\newcommand{\abs}[1]{\left\lvert #1\right\rvert}
\newcommand{\cotan}{\mbox{cotan}}
% Because for some reason it's undefined
\newcommand{\multequal}[4]
{
    \left\{
        \begin{array}{l l}
            #1&#2\\\\
            #3&#4
        \end{array}
    \right.
}
% 1 and 3 are the values, 2 and 4 are the conditions

% ------------------------------Linear Algebra----------------------------------
\let\vaccent=\v % Renaming the funny \v{} accent to \vaccent{}
\renewcommand{\v}[1]{\ensuremath{\mathbf{#1}}}
\newcommand{\gv}[1]{\ensuremath{\mathbf{$#1$}}}
% Alternative representation of vectors, bold vs the arrow
\DeclareMathOperator{\Tr}{Tr}
\newcommand{\trace}[1]{\ensuremath{\Tr\left(#1\right)}}
% Two alternatives for trace, one comes with auto-sized brakets
\newcommand{\ptrace}[2]{\Tr_{#1}\left(#2\right)}
% Partial trace
\newcommand{\uv}[1]{\mathbf{\hat{#1}}} % for unit vector
\newcommand{\I}{\ensuremath{\mathbb{I}}}
\newcommand{\commute}[2]{\left[#1, #2\right]}


% ----------------------------------Quantum-------------------------------------
\newcommand{\bra}[1]{\ensuremath{\left\langle #1\right\rvert}}
\newcommand{\ket}[1]{\ensuremath{\left\lvert #1\right\rangle}}
\newcommand{\braket}[2]{\ensuremath{\left\langle #1\right\lvert\left.
            #2\right\rangle}}
\newcommand{\ketbra}[2]{\ensuremath{\left\lvert
            #1\right\rangle\left\langle #2\right\rvert}}
\newcommand{\hammy}{\ensuremath{\hat{H}}} % Hamiltonian
\newcommand{\rhohat}{\hat{\rho}}
\newcommand{\shat}[1]{\hat{S}_{#1}}
\newcommand{\lhat}[1]{hat{L}_{#1}}


% ----------------------------------Calculus------------------------------------
\newcommand{\del}{\patrial} % Because typing \patrial is long
\newcommand{\dbar}{\mathchar'26\mkern-10mud}
\let\underdot=\d
\renewcommand{\d}[2]{\frac{d #1}{d #2}}
\newcommand{\dd}[2]{\frac{d^2 #1}{d #2^2}}
\newcommand{\pd}[2]{\frac{\partial #1}{\partial #2}}
\newcommand{\pdd}[2]{\frac{\partial^2 #1}{\partial #2^2}}
\newcommand{\pdc}[3]{\frac{\partial^2 #1}{\partial #2\partial #3}}
\newcommand{\grad}[1]{\gv{\nabla} #1}
\newcommand{\curl}[1]{\gv{\nabla}\times #1}
\newcommand{\dive}[1]{\gv{\nabla}\cdot #1}
% I actually use the \div symbol, so I didn't name it \div
\newcommand{\indint}[4]{\displaystyle\int_{#1}^{#2}#3\ d#4}
\newcommand{\inteval}[3]{#1\bigg\rvert_{#2}^{#3}}
\newcommand{\deval}[2]{#1\bigg\rvert_{#2}}

% --------------------------Electricity and Magnetism--------------------------
\newcommand{\hfield}{\v{H}}
\newcommand{\bfield}{\v{B}}
\newcommand{\efield}{\v{E}}
\newcommand{\econstant}{\epsilon_0}
\newcommand{\mconstant}{\mu_0}

% ---------------------------Theorem Stuff--------------------------------------

\newtheorem{prop}{Proposition}
\newtheorem{thm}{Theorem}[section]
\newtheorem{lem}[thm]{Lemma}
\theoremstyle{definition}
\newtheorem{dfn}{Definition}
\theoremstyle{remark}
\newtheorem*{rmk}{Remark}