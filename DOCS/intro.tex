%% ---------------------------------------------------------------------------
%% --------------Jamob The Elpehant's Custom Physics Header--------------------
%% ----------------------------------------------------------------------------
%% Version 0.1

% Use %% ---------------------------------------------------------------------------
%% --------------Jamob The Elpehant's Custom Physics Header--------------------
%% ----------------------------------------------------------------------------
%% Version 0.1

% Use %% ---------------------------------------------------------------------------
%% --------------Jamob The Elpehant's Custom Physics Header--------------------
%% ----------------------------------------------------------------------------
%% Version 0.1

% Use \input{header.tex} at the top of the file to use this
% Comment in and out as you may see fit


%% --------------------General Stuff------------------------------------------
\documentclass[11pt]{article}
\usepackage{amsmath}  % AMSMath Package
\usepackage{amsthm}   % Theorem Formatting
\usepackage{amssymb}  % Math symbols like \mathbb
\usepackage{graphics} % EPS images
\usepackage{multicol} % Multiple columns in tables
\usepackage{multirow} % Multiple rows in tables
% Sets margins to something reasonable
\usepackage{cancel} % To cross out stuff
\usepackage{enumerate}
\renewcommand{\labelitemii}{(\alph{enumi})}
% Use letters for 2nd level enumeration
\usepackage{fancyhdr}
\newcommand{\header}[4]{\pagestyle{fancy}\lhead{#1\\#2}\rhead{#3\\#4}}

\usepackage{mathtools} % TO BOX ACROSS ALIGNMENTS TABS!!!111 :D

% -----------------------------General Math------------------------------------
\newcommand{\lrbrace}[1]{\left(#1\right)}
\newcommand{\lrbraces}[1]{\left[#1\right]}
% Auto-sized braces because I'm lazy
\let\oldepsilon=\epsilon
\let\oldphi=\phi
\renewcommand{\epsilon}{\varepsilon}
\renewcommand{\phi}{\varphi}
% Renaming epsilon and phi to the cool epsilon and phi
\usepackage{calligra}
\DeclareMathAlphabet{\mathcalligra}{T1}{calligra}{m}{n}
\DeclareFontShape{T1}{calligra}{m}{n}{<->s*[2.2]callig15}{}
\newcommand{\scripty}[1]{\ensuremath{\mathcalligra{#1}}}
% For scripted letters, like \scirpty{r}
\newcommand{\avg}[1]{\left\langle #1\right\rangle}
\newcommand{\abs}[1]{\left\lvert #1\right\rvert}
\newcommand{\cotan}{\mbox{cotan}}
% Because for some reason it's undefined
\newcommand{\multequal}[4]
{
    \left\{
        \begin{array}{l l}
            #1&#2\\\\
            #3&#4
        \end{array}
    \right.
}
% 1 and 3 are the values, 2 and 4 are the conditions

% ------------------------------Linear Algebra----------------------------------
\let\vaccent=\v % Renaming the funny \v{} accent to \vaccent{}
\renewcommand{\v}[1]{\ensuremath{\mathbf{#1}}}
\newcommand{\gv}[1]{\ensuremath{\mathbf{#1}}}
% Alternative representation of vectors, bold vs the arrow
\DeclareMathOperator{\Tr}{Tr}
\newcommand{\trace}[1]{\ensuremath{\Tr\left(#1\right)}}
% Two alternatives for trace, one comes with auto-sized brakets
\newcommand{\ptrace}[2]{\Tr_{#1}\left(#2\right)}
% Partial trace
\newcommand{\uv}[1]{\mathbf{\hat{#1}}} % for unit vector
\newcommand{\I}{\ensuremath{\mathbb{I}}}
\newcommand{\commute}[2]{\left[#1, #2\right]}


% ----------------------------------Quantum-------------------------------------
\newcommand{\bra}[1]{\ensuremath{\left\langle #1\right\rvert}}
\newcommand{\ket}[1]{\ensuremath{\left\lvert #1\right\rangle}}
\newcommand{\braket}[2]{\ensuremath{\left\langle #1\right\lvert\left.
            #2\right\rangle}}
\newcommand{\ketbra}[2]{\ensuremath{\left\lvert
            #1\right\rangle\left\langle #2\right\rvert}}
\newcommand{\hammy}{\ensuremath{\hat{H}}} % Hamiltonian
\newcommand{\rhohat}{\hat{\rho}}
\newcommand{\shat}[1]{\hat{S}_{#1}}
\newcommand{\lhat}[1]{hat{L}_{#1}}


% ----------------------------------Calculus------------------------------------
\newcommand{\del}{\patrial} % Because typing \patrial is long
\newcommand{\dbar}{\mathchar'26\mkern-10mud}
\let\underdot=\d
\renewcommand{\d}[2]{\frac{d #1}{d #2}}
\newcommand{\dd}[2]{\frac{d^2 #1}{d #2^2}}
\newcommand{\pd}[2]{\frac{\partial #1}{\partial #2}}
\newcommand{\pdd}[2]{\frac{\partial^2 #1}{\partial #2^2}}
\newcommand{\pdc}[3]{\frac{\partial^2 #1}{\partial #2\partial #3}}
\newcommand{\grad}[1]{\gv{\nabla} #1}
\newcommand{\curl}[1]{\gv{\nabla}\times \gv{#1}}
\newcommand{\dive}[1]{\gv{\nabla}\cdot \gv{#1}}
% I actually use the \div symbol, so I didn't name it \div
\newcommand{\indint}[4]{\displaystyle\int_{#1}^{#2}#3\ d#4}
\newcommand{\inteval}[3]{#1\bigg\rvert_{#2}^{#3}}
\newcommand{\deval}[2]{#1\bigg\rvert_{#2}}

% --------------------------Electricity and Magnetism--------------------------
\newcommand{\hfield}{\v{H}}
\newcommand{\bfield}{\v{B}}
\newcommand{\efield}{\v{E}}
\newcommand{\econstant}{\epsilon_0}
\newcommand{\mconstant}{\mu_0}

% ---------------------------Theorem Stuff--------------------------------------

\newtheorem{prop}{Proposition}
\newtheorem{thm}{Theorem}[section]
\newtheorem{lem}[thm]{Lemma}
\theoremstyle{definition}
\newtheorem{dfn}{Definition}
\theoremstyle{remark}
\newtheorem*{rmk}{Remark}

\newcounter{definition}
\newenvironment{definition}[1][]{
    \refstepcounter{definition}\par\medskip\noindent%
    \textbf{Definition~\thesection.\thedefinition.} \emph{#1} \rmfamily}{\medskip}

\newcounter{algorithm}
\newenvironment{algorithm}[1][]{
    \refstepcounter{algorithm}\par\medskip\noindent%
    \textbf{Algorithm~\thesection.\thealgorithm.} \emph{#1}
    \rmfamily}{\medskip}

\newcounter{example}
\newenvironment{example}[1][]{\refstepcounter{example}\par\medskip\noindent%
  \textbf{Example~\thesection.\theexample. #1}\rmfamily}{\medskip}

\newcounter{lemma}
\newenvironment{lemma}[1][]{\refstepcounter{lemma}\par\medskip\noindent%
  \textbf{Lemma~\thesection.\thelemma. #1}\rmfamily}{\medskip}
 at the top of the file to use this
% Comment in and out as you may see fit


%% --------------------General Stuff------------------------------------------
\documentclass[11pt]{article}
\usepackage{amsmath}  % AMSMath Package
\usepackage{amsthm}   % Theorem Formatting
\usepackage{amssymb}  % Math symbols like \mathbb
\usepackage{graphics} % EPS images
\usepackage{multicol} % Multiple columns in tables
\usepackage{multirow} % Multiple rows in tables
% Sets margins to something reasonable
\usepackage{cancel} % To cross out stuff
\usepackage{enumerate}
\renewcommand{\labelitemii}{(\alph{enumi})}
% Use letters for 2nd level enumeration
\usepackage{fancyhdr}
\newcommand{\header}[4]{\pagestyle{fancy}\lhead{#1\\#2}\rhead{#3\\#4}}

\usepackage{mathtools} % TO BOX ACROSS ALIGNMENTS TABS!!!111 :D

% -----------------------------General Math------------------------------------
\newcommand{\lrbrace}[1]{\left(#1\right)}
\newcommand{\lrbraces}[1]{\left[#1\right]}
% Auto-sized braces because I'm lazy
\let\oldepsilon=\epsilon
\let\oldphi=\phi
\renewcommand{\epsilon}{\varepsilon}
\renewcommand{\phi}{\varphi}
% Renaming epsilon and phi to the cool epsilon and phi
\usepackage{calligra}
\DeclareMathAlphabet{\mathcalligra}{T1}{calligra}{m}{n}
\DeclareFontShape{T1}{calligra}{m}{n}{<->s*[2.2]callig15}{}
\newcommand{\scripty}[1]{\ensuremath{\mathcalligra{#1}}}
% For scripted letters, like \scirpty{r}
\newcommand{\avg}[1]{\left\langle #1\right\rangle}
\newcommand{\abs}[1]{\left\lvert #1\right\rvert}
\newcommand{\cotan}{\mbox{cotan}}
% Because for some reason it's undefined
\newcommand{\multequal}[4]
{
    \left\{
        \begin{array}{l l}
            #1&#2\\\\
            #3&#4
        \end{array}
    \right.
}
% 1 and 3 are the values, 2 and 4 are the conditions

% ------------------------------Linear Algebra----------------------------------
\let\vaccent=\v % Renaming the funny \v{} accent to \vaccent{}
\renewcommand{\v}[1]{\ensuremath{\mathbf{#1}}}
\newcommand{\gv}[1]{\ensuremath{\mathbf{#1}}}
% Alternative representation of vectors, bold vs the arrow
\DeclareMathOperator{\Tr}{Tr}
\newcommand{\trace}[1]{\ensuremath{\Tr\left(#1\right)}}
% Two alternatives for trace, one comes with auto-sized brakets
\newcommand{\ptrace}[2]{\Tr_{#1}\left(#2\right)}
% Partial trace
\newcommand{\uv}[1]{\mathbf{\hat{#1}}} % for unit vector
\newcommand{\I}{\ensuremath{\mathbb{I}}}
\newcommand{\commute}[2]{\left[#1, #2\right]}


% ----------------------------------Quantum-------------------------------------
\newcommand{\bra}[1]{\ensuremath{\left\langle #1\right\rvert}}
\newcommand{\ket}[1]{\ensuremath{\left\lvert #1\right\rangle}}
\newcommand{\braket}[2]{\ensuremath{\left\langle #1\right\lvert\left.
            #2\right\rangle}}
\newcommand{\ketbra}[2]{\ensuremath{\left\lvert
            #1\right\rangle\left\langle #2\right\rvert}}
\newcommand{\hammy}{\ensuremath{\hat{H}}} % Hamiltonian
\newcommand{\rhohat}{\hat{\rho}}
\newcommand{\shat}[1]{\hat{S}_{#1}}
\newcommand{\lhat}[1]{hat{L}_{#1}}


% ----------------------------------Calculus------------------------------------
\newcommand{\del}{\patrial} % Because typing \patrial is long
\newcommand{\dbar}{\mathchar'26\mkern-10mud}
\let\underdot=\d
\renewcommand{\d}[2]{\frac{d #1}{d #2}}
\newcommand{\dd}[2]{\frac{d^2 #1}{d #2^2}}
\newcommand{\pd}[2]{\frac{\partial #1}{\partial #2}}
\newcommand{\pdd}[2]{\frac{\partial^2 #1}{\partial #2^2}}
\newcommand{\pdc}[3]{\frac{\partial^2 #1}{\partial #2\partial #3}}
\newcommand{\grad}[1]{\gv{\nabla} #1}
\newcommand{\curl}[1]{\gv{\nabla}\times \gv{#1}}
\newcommand{\dive}[1]{\gv{\nabla}\cdot \gv{#1}}
% I actually use the \div symbol, so I didn't name it \div
\newcommand{\indint}[4]{\displaystyle\int_{#1}^{#2}#3\ d#4}
\newcommand{\inteval}[3]{#1\bigg\rvert_{#2}^{#3}}
\newcommand{\deval}[2]{#1\bigg\rvert_{#2}}

% --------------------------Electricity and Magnetism--------------------------
\newcommand{\hfield}{\v{H}}
\newcommand{\bfield}{\v{B}}
\newcommand{\efield}{\v{E}}
\newcommand{\econstant}{\epsilon_0}
\newcommand{\mconstant}{\mu_0}

% ---------------------------Theorem Stuff--------------------------------------

\newtheorem{prop}{Proposition}
\newtheorem{thm}{Theorem}[section]
\newtheorem{lem}[thm]{Lemma}
\theoremstyle{definition}
\newtheorem{dfn}{Definition}
\theoremstyle{remark}
\newtheorem*{rmk}{Remark}

\newcounter{definition}
\newenvironment{definition}[1][]{
    \refstepcounter{definition}\par\medskip\noindent%
    \textbf{Definition~\thesection.\thedefinition.} \emph{#1} \rmfamily}{\medskip}

\newcounter{algorithm}
\newenvironment{algorithm}[1][]{
    \refstepcounter{algorithm}\par\medskip\noindent%
    \textbf{Algorithm~\thesection.\thealgorithm.} \emph{#1}
    \rmfamily}{\medskip}

\newcounter{example}
\newenvironment{example}[1][]{\refstepcounter{example}\par\medskip\noindent%
  \textbf{Example~\thesection.\theexample. #1}\rmfamily}{\medskip}

\newcounter{lemma}
\newenvironment{lemma}[1][]{\refstepcounter{lemma}\par\medskip\noindent%
  \textbf{Lemma~\thesection.\thelemma. #1}\rmfamily}{\medskip}
 at the top of the file to use this
% Comment in and out as you may see fit


%% --------------------General Stuff------------------------------------------
\documentclass[11pt]{article}
\usepackage{amsmath}  % AMSMath Package
\usepackage{amsthm}   % Theorem Formatting
\usepackage{amssymb}  % Math symbols like \mathbb
\usepackage{graphics} % EPS images
\usepackage{multicol} % Multiple columns in tables
\usepackage{multirow} % Multiple rows in tables
% Sets margins to something reasonable
\usepackage{cancel} % To cross out stuff
\usepackage{enumerate}
\renewcommand{\labelitemii}{(\alph{enumi})}
% Use letters for 2nd level enumeration
\usepackage{fancyhdr}
\newcommand{\header}[4]{\pagestyle{fancy}\lhead{#1\\#2}\rhead{#3\\#4}}

\usepackage{mathtools} % TO BOX ACROSS ALIGNMENTS TABS!!!111 :D

% -----------------------------General Math------------------------------------
\newcommand{\lrbrace}[1]{\left(#1\right)}
\newcommand{\lrbraces}[1]{\left[#1\right]}
% Auto-sized braces because I'm lazy
\let\oldepsilon=\epsilon
\let\oldphi=\phi
\renewcommand{\epsilon}{\varepsilon}
\renewcommand{\phi}{\varphi}
% Renaming epsilon and phi to the cool epsilon and phi
\usepackage{calligra}
\DeclareMathAlphabet{\mathcalligra}{T1}{calligra}{m}{n}
\DeclareFontShape{T1}{calligra}{m}{n}{<->s*[2.2]callig15}{}
\newcommand{\scripty}[1]{\ensuremath{\mathcalligra{#1}}}
% For scripted letters, like \scirpty{r}
\newcommand{\avg}[1]{\left\langle #1\right\rangle}
\newcommand{\abs}[1]{\left\lvert #1\right\rvert}
\newcommand{\cotan}{\mbox{cotan}}
% Because for some reason it's undefined
\newcommand{\multequal}[4]
{
    \left\{
        \begin{array}{l l}
            #1&#2\\\\
            #3&#4
        \end{array}
    \right.
}
% 1 and 3 are the values, 2 and 4 are the conditions

% ------------------------------Linear Algebra----------------------------------
\let\vaccent=\v % Renaming the funny \v{} accent to \vaccent{}
\renewcommand{\v}[1]{\ensuremath{\mathbf{#1}}}
\newcommand{\gv}[1]{\ensuremath{\mathbf{#1}}}
% Alternative representation of vectors, bold vs the arrow
\DeclareMathOperator{\Tr}{Tr}
\newcommand{\trace}[1]{\ensuremath{\Tr\left(#1\right)}}
% Two alternatives for trace, one comes with auto-sized brakets
\newcommand{\ptrace}[2]{\Tr_{#1}\left(#2\right)}
% Partial trace
\newcommand{\uv}[1]{\mathbf{\hat{#1}}} % for unit vector
\newcommand{\I}{\ensuremath{\mathbb{I}}}
\newcommand{\commute}[2]{\left[#1, #2\right]}


% ----------------------------------Quantum-------------------------------------
\newcommand{\bra}[1]{\ensuremath{\left\langle #1\right\rvert}}
\newcommand{\ket}[1]{\ensuremath{\left\lvert #1\right\rangle}}
\newcommand{\braket}[2]{\ensuremath{\left\langle #1\right\lvert\left.
            #2\right\rangle}}
\newcommand{\ketbra}[2]{\ensuremath{\left\lvert
            #1\right\rangle\left\langle #2\right\rvert}}
\newcommand{\hammy}{\ensuremath{\hat{H}}} % Hamiltonian
\newcommand{\rhohat}{\hat{\rho}}
\newcommand{\shat}[1]{\hat{S}_{#1}}
\newcommand{\lhat}[1]{hat{L}_{#1}}


% ----------------------------------Calculus------------------------------------
\newcommand{\del}{\patrial} % Because typing \patrial is long
\newcommand{\dbar}{\mathchar'26\mkern-10mud}
\let\underdot=\d
\renewcommand{\d}[2]{\frac{d #1}{d #2}}
\newcommand{\dd}[2]{\frac{d^2 #1}{d #2^2}}
\newcommand{\pd}[2]{\frac{\partial #1}{\partial #2}}
\newcommand{\pdd}[2]{\frac{\partial^2 #1}{\partial #2^2}}
\newcommand{\pdc}[3]{\frac{\partial^2 #1}{\partial #2\partial #3}}
\newcommand{\grad}[1]{\gv{\nabla} #1}
\newcommand{\curl}[1]{\gv{\nabla}\times \gv{#1}}
\newcommand{\dive}[1]{\gv{\nabla}\cdot \gv{#1}}
% I actually use the \div symbol, so I didn't name it \div
\newcommand{\indint}[4]{\displaystyle\int_{#1}^{#2}#3\ d#4}
\newcommand{\inteval}[3]{#1\bigg\rvert_{#2}^{#3}}
\newcommand{\deval}[2]{#1\bigg\rvert_{#2}}

% --------------------------Electricity and Magnetism--------------------------
\newcommand{\hfield}{\v{H}}
\newcommand{\bfield}{\v{B}}
\newcommand{\efield}{\v{E}}
\newcommand{\econstant}{\epsilon_0}
\newcommand{\mconstant}{\mu_0}

% ---------------------------Theorem Stuff--------------------------------------

\newtheorem{prop}{Proposition}
\newtheorem{thm}{Theorem}[section]
\newtheorem{lem}[thm]{Lemma}
\theoremstyle{definition}
\newtheorem{dfn}{Definition}
\theoremstyle{remark}
\newtheorem*{rmk}{Remark}

\newcounter{definition}
\newenvironment{definition}[1][]{
    \refstepcounter{definition}\par\medskip\noindent%
    \textbf{Definition~\thesection.\thedefinition.} \emph{#1} \rmfamily}{\medskip}

\newcounter{algorithm}
\newenvironment{algorithm}[1][]{
    \refstepcounter{algorithm}\par\medskip\noindent%
    \textbf{Algorithm~\thesection.\thealgorithm.} \emph{#1}
    \rmfamily}{\medskip}

\newcounter{example}
\newenvironment{example}[1][]{\refstepcounter{example}\par\medskip\noindent%
  \textbf{Example~\thesection.\theexample. #1}\rmfamily}{\medskip}

\newcounter{lemma}
\newenvironment{lemma}[1][]{\refstepcounter{lemma}\par\medskip\noindent%
  \textbf{Lemma~\thesection.\thelemma. #1}\rmfamily}{\medskip}

\title{HW6}
\usepackage{qtree}
\usepackage[mathscr]{euscript}
\usepackage{mathrsfs}
\title{Google Summer of Code\\
    Numerically Solving PDEs in Scala}
\author{Timothy Spurdle}
\date{}
\begin{document}
\maketitle
\begin{abstract}
    Although multiple methods exist to numerically solve
    PDEs. Depending on the boundary conditions, not all of them will
    always work. The method which can be adjusted to work in the most
    cases is the finite difference method. Although it is an extremely
    simple method, many subtlties arrise when deciding on the step
    size and when calculating errors. 
\end{abstract}
\section{Introduction}
Partial differential equations are used to model many of our world
phenomenons. From the heat equation to Laplace's equation, all types
of PDEs exist and many fields of science use them to model specific
situations. Often, one can not actually find a real solution for a
PDE, normally because of the complex nature of it. Despite that,
numerical solutions can almost always be used.

\subsection{Basic Interface}
I would like to implement some of these methods as well as error
propagation code. The first thing I will be working on will be a
parser, so that one can write PDE's in the more natural way. Scala
offers a good library for parsing strings into tokens. I'm thinking of
having the PDE written as something like
\begin{verbatim}
    val eq = PDE("a u_{xx} + b u_{tx} + c u_{tt} + d u_t+ e u_x + f u = g(x, t)")
\end{verbatim}
Where $a, b, c, d, e, f$ and $g$ are functions of $x$ and $t$.

This follows \LaTeX's convention, so potentially one could even adjoin
my project to a \LaTeX/Scala PDE solver/generator. Unless I can think
of a more natural way to express PDEs, I'll keep it at this. 

The above function the gets converted to something like:
\begin{verbatim}
    class LinearPDE2 (a: (Double, Double) => Double,
                     b: (Double, Double) => Double, 
                     c: (Double, Double) => Double,
                     d: (Double, Double) => Double,
                     e: (Double, Double) => Double,
                     f: (Double, Double) => Double,
                     g: (Double, Double) => Double)
\end{verbatim}

At which point to generate a solution, one calls either
\begin{verbatim}
    val (u, delta) = eq.generate_solution(boundary)
\end{verbatim}
to generate a function $u$ which is the solution and $\delta$ the
error on the approximation. Although $u$ is a function, the solution
is stored as a matrix of solution points. When one calls say $u(x, y)$
unless $x$ and $y$ are points at exact step sizes, a weighted average
of nearby points is returned. To call the function,
one needs to define boundary conditions. In cartesian coordinates,
that would be at least 3 functions which define an enclosing area. I
will start by doing it for rectangular boundaries and then move on to
other shapes. A rectangular boundary is defined by
\begin{verbatim}
    class Boundary (
        val b1: ((Double, Double) => Double, (Double, Double), (Int, Double)),
        val b2: ((Double, Double) => Double, (Double, Double), (Int, Double)), 
        val b3:  ((Double, Double) => Double, (Double, Double), (Int, Double)),
        val b4:  ((Double, Double) => Double, (Double, Double), (Int, Double))
        )
\end{verbatim}
For example,
\begin{verbatim}
    ((x: Double, t: Double) => x*t, (-1, 1), (1, 3))
\end{verbatim}
would represent
\begin{equation}
    u(x, 3) = g(x) = x*3\ for\ -1\leq x\leq 1
\end{equation}
So the first input is a function. The second is the interval from
which it is valid and the third is to represent what variable is held
constant and at what point. That notation is confusing, but I have not
thought of any other solution for the while. Once the boundary is
created, assertions are made to ensure that the connecting points
match.

When calling the function which generates a solution, derivatives of
the boundary are computed to ensure that the boundary is viable for
the PDE.

\subsection{Inheritance}
I have yet to find any place where inheritance could save a large
amount of code, but for expansion purposes, some basic inheritance
models will look like:

% \Tree[.PDE [.First Order Linear Non-Linear] test]
\Tree [.PDE  [.Linear LinearPDE1 LinearPDE2 ] [.Non-Linear
NonLinearPDE1 NonLinearPDE2 ]]
\\
\\

\Tree [.Boundary  [.Cartesian 3-sides 4-sides 5+sides ] Spherical
Cylindrical ]

\subsection{Lightweight Modular Staging}
Scala's Lightweight Modular Staging(LMS) would make the code look
something like:
\begin{verbatim}
trait Equation extends Base{
    abstract class Expr[T]
    case class Funct[T](fn: Function2[Double, Double, Double]) extends Expr[T]
    case class fnName[T](x: String) extends Expr[T]
    case class Const[T](x: T) extends Expr[T]
    case class Sym[T](n: Double) extends Expr[T]

    abstract class Def[T]

    def findOrCreateDefinition[T](rhs: Def[T]): Sym[T]
    implicitdeftoExp[T](d: Def[T]): Exp[T] = findOrCreateDefinition(d)
}

trait LinearPDE2 extends Base {
    object LinearPDE2 {
        def apply[T](a: Rep[Function2[Double, Double, Double]],
                     b: Rep[Function2[Double, Double, Double]], 
                     c: Rep[Function2[Double, Double, Double]],
                     d: Rep[Function2[Double, Double, Double]],
                     e: Rep[Function2[Double, Double, Double]],
                     f: Rep[Function2[Double, Double, Double]],
                     g: Rep[Function2[Double, Double, Double]]) = PDE_new(a, b, c, d, e, f, g)
    }

    def PDE_new(a: Rep[Function2[Double, Double, Double]],
                b: Rep[Function2[Double, Double, Double]], 
                c: Rep[Function2[Double, Double, Double]],
                d: Rep[Function2[Double, Double, Double]],
                e: Rep[Function2[Double, Double, Double]],
                f: Rep[Function2[Double, Double, Double]],
                g: Rep[Function2[Double, Double, Double]]): Rep[LinearPDE2]
}

trait LinearPDE2OpsExp extends BasePDE {
    case class PDENew(a: Expr[Double],
                         b: Expr[Double],
                         c: Expr[Double],
                         d: Expr[Double],
                         e: Expr[Double],
                         f: Expr[Double],
                         g: Expr[Double])
      extends Def[LinearPDE2]

    def pde_new(a: Expr[Double],
                         b: Expr[Double],
                         c: Expr[Double],
                         d: Expr[Double],
                         e: Expr[Double],
                         f: Expr[Double],
                         g: Expr[Double]): Expr[PDE] = PDENew(a, b, c, d, e, f, g)

}


\end{verbatim}
Where the above code is for 2nd order linear PDEs specifically. After
that one will need to translate methods into nodes
representing a PDE. Something like

\begin{verbatim}
trait LinearPDE2 extends Base {
    object LinearPDE2 {
        def apply[T](a: Rep[Function2[Double, Double, Double] ],
                     b: Rep[Function2[Double, Double, Double] ], 
                     c: Rep[Function2[Double, Double, Double] ],
                     d: Rep[Function2[Double, Double, Double] ],
                     e: Rep[Function2[Double, Double, Double] ],
                     f: Rep[Function2[Double, Double, Double] ],
                     g: Rep[Function2[Double, Double, Double] ])
                       = PDE_new(a, b, c, d, e, f, g)
    }
    def generate_solution (V: Rep[Boundary]) = PDE_solve(V)

    def PDE_new(a: Rep[Function2[Double, Double, Double] ],
                b: Rep[Function2[Double, Double, Double] ], 
                c: Rep[Function2[Double, Double, Double] ],
                d: Rep[Function2[Double, Double, Double] ],
                e: Rep[Function2[Double, Double, Double] ],
                f: Rep[Function2[Double, Double, Double] ],
                g: Rep[Function2[Double, Double, Double] ]): Rep[LinearPDE2]
    def PDE_solve(V: Rep[Boundary]): Function2[Double, Double, Double] 
}

trait LinearPDE2OpsExp extends BasePDE {
    case class PDENew(a: Expr[Double],
                      b: Expr[Double],
                      c: Expr[Double],
                      d: Expr[Double],
                      e: Expr[Double],
                      f: Expr[Double],
                      g: Expr[Double])
      extends Def[LinearPDE2]

    def pde_new(a: Expr[Double],
                b: Expr[Double],
                c: Expr[Double],
                d: Expr[Double],
                e: Expr[Double],
                f: Expr[Double],
                g: Expr[Double]): Expr[PDE] = PDENew(a, b, c, d, e, f, g)
}
\end{verbatim}
\section{Theory}
\emph{This section is incomplete and will be build up gradually.}

\subsection{List of sections used from Leveque}
\begin{itemize}
    
\item[1.] All of section 1
\item[2.] .2-.9
\item[3.] .3
\item[6.] .1-.4
\item[7.] Most of 7 
\end{itemize}
\subsection{Fourier Analysis}
This method only applies to linear PDEs, but it has the advantage of
being relatively simple and producing exact solutions up to rounding
errors.

Page 150[158] of Leveque's notes.  

\section{Schedule/Milestones}
\begin{itemize}
\item[July $8^{st}$:] Completed the LMS structure for Linear 1st
    and 2nd order PDEs. 
\item[July $22^{th}$:] Solutions for $1^{st}$ and $2^{nd}$ order
    PDEs can now be generated. Code will first be written for linear
    PDEs using Fourier Analysis, then for finite difference method for
    elliptical equations and finally method of lines.  
    
    \emph{Writeup of documentation for the GSoCs midterm evaluation
        will be done before the 15th.}
    
\item[Aug $5^{th}$:] Added code in traits LinearPDE* and
    LinearPDE*OpsExp to take into account error generation.
    
\item[Aug $19^{th}$:] Adjust computation methods for the stiffness of
    the equation. 
    
\item[Sept $2^{nd}$:] Finalize documentation and code for GSoC final evaluations.
\end{itemize}

\end{document}
